\documentclass[letterpaper]{article}
\usepackage[pass,showframe]{geometry}

\usepackage{ijcai17}
\usepackage{times}
\usepackage{graphicx}

\newcommand{\citet}[1]{\citeauthor{#1} \shortcite{#1}}
\newcommand{\citep}[1]{\cite{#1}}

\title{A Partitioning Algorithm for Maximum Common (Connected) Subgraphs\thanks{This work
was supported by the Engineering and Physical Sciences Research Council [grant
numbers EP/K503058/1 and EP/M508056/1]}}
\author{Ciaran McCreesh \and Patrick Prosser \and James Trimble \\
University of Glasgow, Glasgow, Scotland \\
j.trimble.1@research.gla.ac.uk}

\begin{document}

\maketitle

\begin{abstract}
    James came up with an awesome new algorithm for the maximum common (connected)
    subgraph problem.
\end{abstract}

\section{Introduction}

\citet{UpcomingAAAIPaper}

\citet{DBLP:conf/cp/McCreeshNPS16}

\citet{DBLP:conf/cp/NdiayeS11}

\section{Results}

\begin{figure*}[tb]
    \centering
    \includegraphics*{gen-graph-mcs.pdf}
    \caption{First plot is induced SIP instances, second plot is unlabelled
    undirected MCS instances.}\label{figure:runtimes}
\end{figure*}

\section{Connected}

\section{Parallel}

\section{Labelled and Directed Edges}

James reckons that finding a match is so rare that it's more or less down to
finding a maximum independent set, at least in the non-connected case.

\bibliographystyle{named}
\bibliography{paper}

\end{document}

